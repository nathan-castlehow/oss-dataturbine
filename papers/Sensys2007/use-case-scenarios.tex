\section{Use-case scenarios}\label{sec:use-case-scenarios}

Use-case scenarios are essential tools for understanding requirements and ensuring that proposed developments address critical domain needs.  Through our collaborations with domain scientists and engineers across various science communities, we have developed five use-case scenarios to inform the software improvements and the prototype demonstrations.  These scenarios capture core activities that are common across all environmental observing systems and rely on efficient cyberinfrastructure middleware services.  These scenarios are briefly described in the following sections. Section ~\ref{sec:rbnb-features} then explains how the DataTurbine addresses the requirements.

\emph{Data Acquisition:} The primary objective of an observing system is to capture and annotate field measurements, including data and image streams.  This requires the ability to collect data from deployed devices, annotate the data with appropriate metadata, and move the data reliably and efficiently from observation site to data collection center.  The data acquisition scenario captures the system requirements for this process. 

\emph{Discovery and Real-time Access to Data Streams from Remote Instruments:} Many observing system functions require real-time or near-real-time operations on data streams (in addition to data archives).  This is especially true for event detection and response, QA/QC, virtual observatories, and distributed experimentation.  In these situations the ability to identify, address, and manipulate individual data streams is critical.  This scenario captures these requirements and exercises the stream management functionality of the RBNB DataTurbine system. 

\emph{Data Visualization and Analysis:} Understanding and utilizing real-time observing systems data requires complex visualization and analysis services.  This raises requirements on the streaming data middleware to support efficient integration of custom and commercial tools.  This scenario captures the process of interfacing and utilizing analysis and visualization tools with the RBNB DataTurbine system.  This includes both system integration tasks and data stream analysis and visualization tasks.

\emph{Distributed Resource Sharing and Virtual Observatories:} Constructing a virtual sensor network observatory requires services that allow users to discover and access sensors and sensor streams across multiple observing systems. Under this scenario, a user, once authenticated, can access resources beyond those directly owned by his/her organization. This scenario captures the requirements of providing user authentication and authorization, and secure data transmission. Note that we are not addressing policy issues of distributed resource sharing in this scenario.  We are addressing the cyberinfrastructure services that are required to implement and enforce such policies.

\emph{Large-scale Distributed Collaborative Experiments:} Large-scale observing systems must support collaborative experiments spanning multiple groups and physical laboratories.  Remote real-time collaborations require streaming middleware support to ensure synchronized shared experiences with data, data products, and analysis and visualization tools. The challenges here include reliable data transmission in the face of unreliable networks, scalability to handle hundreds of collaborators, observers and outreach classes, the ability to handle enormous amounts of data with minimal latency.  This scenario captures these requirements and exercises the topology, scalability, single sign-on and robustness of the RBNB DataTurbine system. %In the next section, we describe how RBNB DataTurbine middleware addresses the requirements derived from these scenarios. 
