\subsection{RBNB DataTurbine Background}\label{sec:rbnb-background}

Environmental observing systems want to have access to real-time data for a variety of reasons. Use cases include adaptive sampling rates, failure detection and correction, quality assurance and simple observation. In addition, real-time data access can be used to generate interest and buy-in from various stakeholders, especially the public.

It should be noted that our use of the 'real-time' description is more properly soft real-time, in that guarantees are not made with respect to latency or response. In these applications over routed networks, hard real-time is judged to be not useful.

%More specifically, to initiate a timely response, including adaptive sampling (different sampling rates, different instruments), to catch system failures so that remediation can ensure quality data (fix the system before it generates bad data), and to send a human observer into the field (for observations, recovery of samples from a transient event, recover specimens, e.g., dead animals). In addition, real-time data access can be used to generate interest and buy-in from various stakeholders, especially the public.
%  since real-time observations are very interesting to the general public. 

Real-time streaming data is a natural model for many of the applications in observing systems, in particular event detection and pattern recognition.  Many of these applications involve filters over data values, or more generally, functions over sliding temporal windows.  By making data streams first-class objects, we have interfaces that support these operations in a natural way.

The management of real-time streaming data in large-scale collaborative applications also presents major processing, communication and administrative challenges. These applications must provide scalable and secure support for data acquisition, instrument and data stream management, and analysis and visualization. Most applications address these issues by building custom systems that are inevitably complex and difficult to support. Extensibility, scalability, and interoperability are often sacrificed under this approach. The goal of RBNB DataTurbine is to address these cyberinfrastructure challenges in a principled manner.

The RBNB DataTurbine middleware provides a modular, scalable, robust environment for meeting these challenges while providing security, configuration management, routing, and data archival services. The RBNB DataTurbine system acts as an intermediary between dissimilar data monitoring and analysis devices and applications. 

As shown in Figure ~\ref{fig:general-arch}, we use a modular architecture, where a source or "feeder" program is a Java application that acquires data from an external source (e.g. video camera, data acquisition (DAQ) system, microphone, etc) and feeds it into the RBNB server. Additional modules display and manipulate data fetched from the RBNB server. This allows flexible configuration where RBNB serves as a coupling between relatively simple and "single purpose" suppliers of data and consumers of data, both of which are presented a logical grouping of physical data sources. RBNB supports the modular addition of new sources and sinks with a clear separation of design, coding, and testing (ref. Figure~\ref{fig:general-arch}). 

From the perspective of distributed systems, the RBNB DataTurbine is a "black box" from which applications and devices send data and receive data. RBNB DataTurbine handles all data management operations between data sources and sinks, including reliable transport, routing, scheduling, and security.  RBNB accomplishes this through the innovative use of memory and file-based ring buffers combined with flexible network objects. Ring buffers are a programmer-configurable mixture of memory and disk, allowing system tuning to meet application-dependent data management requirements. Network bus elements perform data stream multiplexing and routing. These elements combine to support seamless real-time data archiving and distribution over existing local and wide area networks. Ring buffers also connect directly to client applications to provide TiVo-like services including data stream subscription, capture, rewind, and replay. This presents clients with a simple, uniform interface to real-time and historical (playback) data. RBNB's Java implementation language lends it wide platform flexibility, as is further detailed in section ~\ref{sec:rbnb-internals}.

RBNB DataTurbine shares some common features with other existing data management systems; however RBNB DataTurbine is unique in its support for science and engineering applications.  Commercial programs such as MSMQ~\cite{MSMQ} and Websphere MQ~\cite{MQ}, NaradaBrokering ~\cite{Pallickara03} and similar standards including Enterprise messaging systems~\cite{EMS},  Enterprise Service Bus~\cite{Chappell04}, Java Message Service~\cite{JMS}, CORBA~\cite{CORBA}, and various publish-subscribe systems ~\cite{liu-03} provide support for guaranteed messaging, but fail on other science and engineering requirements. These enterprise messaging systems have subtly different requirements that are derived from their focus on business processing models.  For example, science applications require persistence of delivered data, robust metadata annotation, and integration of heterogeneous instruments and data types, e.g., numeric, audio, and video data.  The only other middleware system that approaches RBNB DataTurbine is the Antelope system from Boulder Real Time Systems (BRTT)~\cite{BRTT}.  Antelope is a successful streaming data product that supports science and engineering applications.  It is also built on the concept of a ring buffer.  Compared to RBNB DataTurbine, Antelope presents challenges in procurement and management.  As a commercial product, it can be too expensive for many applications and communities, and the licensing restrictions make it difficult to manage. In our experience, RBNB DataTurbine provides all the core services of the BRTT Antelope, plus other attractive features: it�s open source, written in Java, and it�s easy to install and manage.

RBNB DataTurbine is highly portable and is available for many platforms, from 64-bit multi-core machines and desktop systems to handheld and embedded devices. As a concrete example, we have tested performance of RBNB DataTurbine on a 8 core Sun Fire T2000 Server~\cite{sun-T2000}, Dual-core Linux servers, to Gumstix devices~\cite{gumstix}. To give an idea, a typical Gumstix device has an Intel Xscale - 400 MHz processor with $64$ MB RAM and $16$ MB Flash and runs Linux $2.6$ OS. On the other end, Sun Fire T2000 server has 8 cores (UltraSPARC T1 processor) , 16 GB memory, runs Solaris OS and is connected to a 9 TB storage (RAID). Currently we are working on porting it to cell phones~\cite{nokia-n80} via J2ME.


