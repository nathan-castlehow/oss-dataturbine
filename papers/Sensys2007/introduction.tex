\section{Introduction}\label{sec:introduction}

The % NSF's 
vision of large-scale sensor-based observing systems to address the National Research Council�s �Grand Challenges� for environmental science relies on robust cyberinfrastructure~\cite{NRC01}.  Since the original NRC report in 2001, significant progress has been made in sensor based
observing systems. The Arzberger report, CLEANER commiee, and the national ecological observatory committee among others have all articulated the science, engineering, and educational issues ~\cite{Arzberger04, NRC06, NEON03}. Exploratory projects, small-scale deployments, and large-scale �narrow-themed� observatories have been fielded � with success~\cite{GLEON, USArray, NEES, LEAD, MESONET}.  From these observing systems, there are a number of cyberinfrastructure tools that are now available. Their success and limitations motivate the core requirements for the cyberinfrastructure needed for the next generation of sensor-based observing systems.

The environmental science and engineering communities are actively engaged in planning and developing the next generation of large-scale sensor-based observing systems~\cite{NEON,ORION}.  
These systems face many significant challenges including:  \textbf{heterogeneity of instrumentation} and \textbf{complexity of data stream management}.  Environmental observing systems incorporate instruments across the spectrum of complexity, from temperature sensors to Acoustic Doppler Current Profilers (ADCP) to streaming video cameras.  Managing these instruments and their data streams is a serious challenge. An ideal solution to these challenges is a streaming data middleware providing modularity, flexibility, and control over complex data interactions. 
 
RBNB DataTurbine started as a commercial streaming data product and has a track record of performance in % NSF and NASA 
several large-scale projects~\cite{NEES, GLEON, NASA-NAMMA}. It satisfies a core set of critical infrastructure requirements that are common across a number of observing systems initiatives, including reliable data transport, the promotion of sensors and sensor streams to first-class objects, a framework for the integration of heterogeneous instruments, and a comprehensive suite of services for data management, routing, synchronization, monitoring, and geo-spatial data visualization. It has been tested in a variety of real-world streaming data applications ~\cite{GLEON, NASA-NAMMA,NEES,NEON}.  It facilitates the development of complex distributed streaming data applications, including real-time virtual observatories and telepresence collaboratories~\cite{hubbard-05}. Recently, the RBNB DataTurbine has been released open-source under the Apache 2.0 license~\cite{APACHE2.0}. In this paper, we describe several key features of  open-source RBNB DataTurbine and discuss how these features address the core requirements  and present the results from real-world deployments.
